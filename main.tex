\documentclass{article}
\usepackage[utf8]{inputenc}

\title{On Some Contributions To Complementarity Theory}
%\author{Aritra Dutta}%{\small Supervisors: Dr. Arup Kumar Das }} 
%\author{Aritra Dutta }{\small Supervisor: Dr. Arup Kumar Das}
\author{{Aritra Dutta}\\
{\and} \\
%{\textit{Reg no:}}\\SMATH1112518}\\
%{\and} \\
{\textit{email:}}\\{aritradutta001@gmail.com}\\
{\and}\\
{\textit{Affiliation:}}\\{ Department of Mathematics, Jadavpur University}\\
{\and}\\
{\textit{Supervisor:}} \\
{Dr. Arup Kumar Das} \\ {\and}\\{ SQC \& OR Unit, ISI Kolkata} }
\date{}

\begin{document}

\maketitle
The complementarity problem is identified as a problem in mathematical programming and provides a framework for several optimization problems. The complementary slackness principle holds not only for the linear programming problem; it also holds for  more general programming problems. The complementary slackness principle for the more general programming problems is based on the Karush-Kuhn-Tucker condition of optimality. For linear and quadratic programs, the Karush-Kuhn-Tucker optimality conditions finally reduce to the study of linear complementarity problems (LCP) and this observation was the early motivation for studying the linear complementarity problem. The linear complementarity problem is the problem of finding a complementary pair of nonnegative vectors in a finite dimensional real vector space that satisfies a given system of inequalities.

Several matrix classes are observed in the context of linear complementarity problems due to the characterization of certain properties of the linear complementarity problems, nice features from the viewpoint of algorithms and applications. Many of the results of linear complementarity problems can be stated in terms of the value of a matrix game and principal pivot transform. The linear complementarity problem is extended to the nonlinear complementarity problem. A number of applications of nonlinear complementarity problems are reported in operations research, multiple objective programming problem, control theory, mathematical economics and engineering.

The algorithm presented by Lemke and Howson to compute an equilibrium pair of strategies to a bimatrix game, later extended by Lemke known as Lemke’s algorithm to solve the linear complementarity problem contributed significantly to the development of the linear complementarity theory. Lemke’s algorithm is a pivotal kind of technique to solve LCP$(q, A)$. This algorithm does not solve every instance of the linear complementarity problem and in some instances, the problem may terminate inconclusively without either computing a solution
 to it or showing that no solution to it exists. This observation motivated me to pursue research in the area of complementarity theory.

My research contributions to complementarity theory are primarily on the study of matrix theoretic properties in the context of linear complementarity problems, extending the processability of Lemke’s algorithm and introducing new solution methods in the context of both linear and nonlinear complementarity problems. For developing a new solution method to solve linear and non-linear complementarity problems, the homotopy concept based on topology and the modification of the newton method is considered. The research contributions are summarized in the following paragraphs.

The class of hidden $Z$-matrix in the context of linear complementarity problem is studied to establish certain matrix theoretic characterization of hidden $Z$-matrix. A descent-type interior point method is proposed to compute the solution of the linear complementarity problem LCP$(q, A)$ given that $A$ is a hidden $Z$ matrix and $q$ is a real vector. The class of column competent matrices is revisited to study some matrix theoretic properties of this class and establish some new results on w-uniqueness properties in connection with column competent matrices which are significant in the context of matrix theory as well as algorithms in operations research.

A new class of $K$-type block matrices is introduced including two classes of block matrices namely block triangular $K$-matrices and hidden block triangular $K$-matrices. The purpose of this article is to study the properties of K-type block matrices in the context of the solution of the linear complementarity problem.

The fundamental idea of the homotopy continuation method is to solve a complementarity problem by tracing a certain continuous path that leads to a solution to the problem. To find the solution approach to the linear complementarity problem a new homotopy function as well as a new approach are proposed to trace a homotopy path. The homotopy path approaching the solution is smooth and bounded.

In another work, a new method is introduced to find the solution to the nonlinear complementarity problem using the homotopy approach. An oligopolistic market equilibrium problem is considered to show the method approaching the solution along a smooth and bounded homotopy path. Applying the homotopy method with vector parameter, the equivalency between the nonlinear complementarity problem and the system of nonlinear equations is established.

As an application of the homotopy continuation method to trace the trajectory the suitable homotopy for a two-person zero-sum discounted stochastic ARAT game is introduced.
 
\end{document}










